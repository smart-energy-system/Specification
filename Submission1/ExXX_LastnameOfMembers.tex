% LaTeX Template für Abgaben an der Universität Stuttgart
% Autor: Sandro Speth
% Bei Fragen: Sandro.Speth@studi.informatik.uni-stuttgart.de
%-----------------------------------------------------------
% Hauptmodul des Templates: Hier können andere Dateien eingebunden werden
% oder Inhalte direkt rein geschrieben werden.
% Kompiliere dieses Modul um eine PDF zu erzeugen.

% Dokumentenart. Ersetze 12pt, falls die Schriftgröße anzupassen ist.
\documentclass[12pt]{scrartcl}
% LaTeX Template für Abgaben an der Universität Stuttgart
% Autor: Sandro Speth
% Bei Fragen: Sandro.Speth@studi.informatik.uni-stuttgart.de
%-----------------------------------------------------------
% Modul fuer verwendete Pakete.
% Neue Pakete einfach einfuegen mit dem \usepackage Befehl:
% \usepackage[options]{packagename}
\usepackage[utf8]{inputenc}
\usepackage[T1]{fontenc}
\usepackage[english]{babel}
\usepackage[backend=bibtex]{biblatex}
\usepackage{graphicx}
\usepackage[pdftex,hyperref,dvipsnames]{xcolor}
\usepackage{listings}
\usepackage[a4paper,lmargin={2cm},rmargin={2cm},tmargin={3.5cm},bmargin = {2.5cm},headheight = {4cm}]{geometry}
\usepackage{amsmath,amssymb,amstext,amsthm}
\usepackage[lined,algonl,boxed]{algorithm2e}
\usepackage{tikz}
\usepackage{hyperref}
\usepackage{url}
\usepackage[inline]{enumitem} % Ermöglicht ändern der enum Item Zahlen
\usepackage[headsepline]{scrpage2}
\usepackage{listings} 
\pagestyle{scrheadings} 
\usetikzlibrary{automata,positioning}
\hypersetup{hidelinks,
	colorlinks=true,
	allcolors=black,
	pdfstartview=Fit,
	breaklinks=true}
	
\colorlet{punct}{red!60!black}
\definecolor{background}{HTML}{EEEEEE}
\definecolor{delim}{RGB}{20,105,176}
\colorlet{numb}{magenta!60!black}

\lstdefinelanguage{json}{
    basicstyle=\normalfont\ttfamily,
    numbers=left,
    numberstyle=\scriptsize,
    stepnumber=1,
    numbersep=8pt,
    showstringspaces=false,
    breaklines=true,
    literate=
     *{0}{{{\color{numb}0}}}{1}
      {1}{{{\color{numb}1}}}{1}
      {2}{{{\color{numb}2}}}{1}
      {3}{{{\color{numb}3}}}{1}
      {4}{{{\color{numb}4}}}{1}
      {5}{{{\color{numb}5}}}{1}
      {6}{{{\color{numb}6}}}{1}
      {7}{{{\color{numb}7}}}{1}
      {8}{{{\color{numb}8}}}{1}
      {9}{{{\color{numb}9}}}{1}
      {:}{{{\color{punct}{:}}}}{1}
      {,}{{{\color{punct}{,}}}}{1}
      {\{}{{{\color{delim}{\{}}}}{1}
      {\}}{{{\color{delim}{\}}}}}{1}
      {[}{{{\color{delim}{[}}}}{1}
      {]}{{{\color{delim}{]}}}}{1},
}


% LaTeX Template für Abgaben an der Universität Stuttgart
% Autor: Sandro Speth
% Bei Fragen: Sandro.Speth@studi.informatik.uni-stuttgart.de
%-----------------------------------------------------------
% Modul beinhaltet Befehl fuer Aufgabennummerierung,
% sowie die Header Informationen.

% Überschreibt enumerate Befehl, sodass 1. Ebene Items mit
%\renewcommand{\theenumi}{(\alph{enumi})}
% (a), (b), etc. nummeriert werden.
%\renewcommand{\labelenumi}{\text{\theenumi}}

% Formatierung der Kopfzeile
% \ohead: Setzt rechten Teil der Kopfzeile mit
% Namen und Matrikelnummern aller Bearbeiter
\ohead{Sandro Speth\\
Markus Zilch\\
Dominik Wagner
}
% \chead{} kann mittleren Kopfzeilen Teil sezten
% \ihead: Setzt linken Teil der Kopfzeile mit
% Modulnamen, Semester und Übungsblattnummer
\ihead{Smart Cities and Internet of Things\\
Wintersemester 18/19\\
Specification}

\title{Fachpraktikum Smart Energy Systems
	Specification}
\date{Wintersemester 18/19}
\author{Sandro Speth\\
Markus Zilch\\
Dominik Wagner}

% Beginn des eigentlichen Dokuments
\begin{document}
\maketitle
\section{Introduction}

The traditional power grid is changing more and more over time.
Due to increasing sensititization for the use of renewable and reliable sources of energy instead of nuclear power sources, there is an increasing accomodation of renewable energy.
To fullfill our daily energy need only with such energy sources is quite difficult and needs lot of planning and simulation.
In this work we build a smart energy system to simulate a smart grid.

A smart grid is an energy efficient system with information and communication technology, automation and awareness of energy consumtion.
There are many different actors and technologies which are connected to each other and interoperate to optimate the grid.
%TODO Add "`Introduction to smart energy systems"' smart grid slides definitions

A smart energy systems creates the bridge between a power grid and a resilient and reliable smart grid.
Users can simulate reliable energy sources, as well as different kinds of energy consumers, e.g. homes or offices.
Simulation of distributed energy sources and automation of processes build an energy management system.
Through this microgrids we can possibly rely completly on renewable energy sources in the future.
This can be checked with our smart energy system.

The report is structured as follows.
In section \ref{sec:systemdesign} we present our system design.
First, we give some basic foundations which are relevant for our smart energy system, such as the difference between $kWh$ and $kW$.
Afterwards we present our functional requirements for the smart energy systems as user stories.
From these functional requirements we created our architecture which will be described in this section as well.



\section{System Design}\label{sec:systemdesign}

\subsection{Difference between $kW$ and $kWh$}\label{sec:diffEnergy}
W is a messuring scale for energy applied per timeinstance.
There are different possibilities to describe $W$ in common terms.
A pretty graphic one is the movement of mass.
$1W$ equals $1kg$ of mass moved by 1 meter in one second: $1 \frac{kg*m^2}{s^3}$.
Or in electrical terms: $1W$ equals 1 Ampere of electrical power with a voltage of 1 Volt.
Both of those formulas are equal to a much simpler Term for Watt: $1 W = 1 J/s$.
In simple terms, 1 Watt is the same as one Joule of energy applied over 1 second.
For completeness, $1kW = 1000 W$\cite{Nelson}\cite{Borvon}\cite{SIStandard}.

$Wh$ are the common term for messuring energyconsumption/-production.
$1Wh$ is $1W$ applied continuously over 1 hour.
$1Wh = 1 W * 1h = 1 J/s * 3600s = 3600J$.
For a scientific context the $Wh$ therefore is simply not used, instead the common SI standard $J$ is used.
In comparison, $Wh$ is the total amount of energy used. $W$ is how much energy is used in a specified timeslot (mostly 1 second)\cite{EURichtlinie}\cite{Bundesgesetz}.


\subsection{Difference between consumption and demand}\label{sec:diffconsumptiondemand}
Electricity consumption and electricity demand are two different properties and measured with different measurement units.
The following section contains a description of both and an example at the end.

\subsubsection{Demand}

The demand is the rate of consumption of electricity or mathematical speaking the demand is the derivation of the consumption [1]. %\cite{https://www.stonybrook.edu/commcms/energy/facts/demand}. 
Most of the time the demand is measured in Watt. If you turn on a 100W light bulb, it will demand 100W while it is turned on. At the same time the grid must provide electricity at a rate of 100W. In most cases it is possible to calculate the demand with the following formula [2]. %http://www.iris.uni-stuttgart.de/lehre/eggenberger/eti/index.html
\begin{equation*}
	Demand = Voltage * Current
\end{equation*}
Some customers also have to pay for the demand or peak demand they have because if you have a higher (peak) demand the grid has to support this [1,3].  %https://www.enertiv.com/resources/what-is-peak-demand,https://www.stonybrook.edu/commcms/energy/facts/demand


\subsubsection{Consumption}

It is easier to understand electricity consumption because we are more used to this concept [1]. Many people deal with electricity consumption while paying their electricity bill because most German electricity meter measure only the consumption. %\cite{https://www.stonybrook.edu/commcms/energy/facts/demand}.
The consumption is the amount of electricity used per time unit [1,3]. Most of the time the consumption is measured in kilowatt per hour.
The formula to calculate the consumption is the following [1]. %https://www.stonybrook.edu/commcms/energy/facts/demand.

\begin{equation*}
	Consumption = Demand * Time
\end{equation*}
For example, 5 W LED bulbs turned on for 1h have the consumption of 5 Wh.
\subsubsection{Difference}

The demand is the rate of which we use energy and the consumption is the total energy used for a given time frame [1]. If someone turns on 1 heating unit with a demand of 1kW for 2 hours than the demand during this hours is 1kW, but the consumption is 2kWh. The consumption is the same if two heating units are used for half an hour, but the demand is doubled (2kW). To put it in simple terms the demand is comparable with the speed of a car and the consumption is the distance you drive. The faster you drive the more distance is accumulated over time.\\\\ %https://www.stonybrook.edu/commcms/energy/facts/demand


[1] Consumption Vs. Demand - Stony Brook University https://www.stonybrook.edu/commcms/energy/facts/demand

[2] Leistung des elektrischen Stroms - Prof. Dr. Otto Eggenberger - Universität Stuttgart 
http://www.iris.uni-stuttgart.de/lehre/eggenberger/eti/index.html

[3] What is Peak Demand? - enertiv https://www.enertiv.com/resources/what-is-peak-demand





\subsection{Userstories}\label{sec:userstories}
\begin{enumerate}
\item As a user, I need to create one or more wind turbines in the simulation so that I can calculate the potential energy output.

\item As a user, I need to create one or more photovoltaic panels in the simulation so that I can calculate the potential energy output.

\item As a user, I need to create one or more batteries in the simulation so to save unused energy of in my simulation.

\item As a user, I need to get the charge state of my batteries to know the impact of the energy storage on the grid.

\item As a user, I need to create one or more homes on the demand side in the simulation so that I can simulate some energy consumer.

\item As a user, I need to create one or more commercial buildings on the demand side in the simulation so to simulate some high energy consumer.

\item As a user, I need to get dynamic energy prices calculated from the simulation to determine if I want to sell my produced energy or store it for later use.

\item As a user, I need to use weather data in the simulation to simulate the smart energy system more precise and realistic.

\item As a user, I need to use already saved weather data in the simulation to not be dependent on the availability of the weather service.

\item As a user, I need to generate a forecast for energy generation and demand using the simulation in order to make informed decisions.

\item As a developer, I want to add more supplier modules than wind tubines and photovoltaic pannels to the simulation to improve the smart energy system in the future with further technology due to adding more kind of supplier. 

\item As a developer, I want to add more consumer modules to the simulation to be able to add more kinds of consumers to the simulation. 

\item As a user, I want to be able to create a smart energy system which is independent to an power grid to simulate a reliable smart grid.

\item As a user, I want to get a visual notification if the supply of energy is smaller than the demand of energy to know when more energy supplier are needed.

\item As a demand module, I need to get energy from the (charged) batteries if the provided supply to small for my demand so that I still have enough money. 

\item As a user, I want to model a rechargeable battery so I can store the energy for later usage, if the supply is greater than the demand.

\item As a user, I need the battery to be able to discharge energy if the supply is lower than the demand in order to make my stored energy usable and keep the demand satisfied.

\item As a user, I need the smart grid to be able to manage peaks in the demand in order to smooth the impact on the grid and reduce the likelyhood of power outages.

\item As a user, I need that the demand side consumers feature different load scenarios like home users and commercial users (constant load, occasionally peak loads) in order to make the simulation accurate for real life applications.

\item As a user, I want be able to use the system with my webbrowser so that I can use different platforms to view it and have easy access to the simulated data.

\item As a user, I want to be able to see the energy supply of each individual supply component in order to be able to assess the efficiency of the supplier.

\item As a user, I want to be able to see the energy demand of individual components for efficiency assessment and informed decision making. 

\item As a user, I want to see a summary of energy supply and demand for all components in order to easily assess the current situation.

\item As a user, I want to adjust the demand by postponing the use of devices during peak hours in order to prevent a complete outage of the grid or to react to one-time-only scenarios.

\item As a user, I want the simulation to adapt the demand based on the price per kWh in order to minimize costs of my energy demands and maximize profits of my energy supply.

\end{enumerate}



\subsection{Architecture}
System architecture Diagram

\subsection*{A5}
add weather component

\subsection*{A6}
reliable and responsive system

\subsection*{A7}
three-tier system architecture


\bibliography{bibliography}
\bibliographystyle{plain}
% Ende des Dokuments
\end{document}
