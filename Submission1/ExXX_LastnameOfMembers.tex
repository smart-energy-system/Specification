% LaTeX Template für Abgaben an der Universität Stuttgart
% Autor: Sandro Speth
% Bei Fragen: Sandro.Speth@studi.informatik.uni-stuttgart.de
%-----------------------------------------------------------
% Hauptmodul des Templates: Hier können andere Dateien eingebunden werden
% oder Inhalte direkt rein geschrieben werden.
% Kompiliere dieses Modul um eine PDF zu erzeugen.

% Dokumentenart. Ersetze 12pt, falls die Schriftgröße anzupassen ist.
\documentclass[12pt]{scrartcl}
% LaTeX Template für Abgaben an der Universität Stuttgart
% Autor: Sandro Speth
% Bei Fragen: Sandro.Speth@studi.informatik.uni-stuttgart.de
%-----------------------------------------------------------
% Modul fuer verwendete Pakete.
% Neue Pakete einfach einfuegen mit dem \usepackage Befehl:
% \usepackage[options]{packagename}
\usepackage[utf8]{inputenc}
\usepackage[T1]{fontenc}
\usepackage[english]{babel}
\usepackage[backend=bibtex]{biblatex}
\usepackage{graphicx}
\usepackage[pdftex,hyperref,dvipsnames]{xcolor}
\usepackage{listings}
\usepackage[a4paper,lmargin={2cm},rmargin={2cm},tmargin={3.5cm},bmargin = {2.5cm},headheight = {4cm}]{geometry}
\usepackage{amsmath,amssymb,amstext,amsthm}
\usepackage[lined,algonl,boxed]{algorithm2e}
\usepackage{tikz}
\usepackage{hyperref}
\usepackage{url}
\usepackage[inline]{enumitem} % Ermöglicht ändern der enum Item Zahlen
\usepackage[headsepline]{scrpage2}
\usepackage{listings} 
\pagestyle{scrheadings} 
\usetikzlibrary{automata,positioning}
\hypersetup{hidelinks,
	colorlinks=true,
	allcolors=black,
	pdfstartview=Fit,
	breaklinks=true}
	
\colorlet{punct}{red!60!black}
\definecolor{background}{HTML}{EEEEEE}
\definecolor{delim}{RGB}{20,105,176}
\colorlet{numb}{magenta!60!black}

\lstdefinelanguage{json}{
    basicstyle=\normalfont\ttfamily,
    numbers=left,
    numberstyle=\scriptsize,
    stepnumber=1,
    numbersep=8pt,
    showstringspaces=false,
    breaklines=true,
    literate=
     *{0}{{{\color{numb}0}}}{1}
      {1}{{{\color{numb}1}}}{1}
      {2}{{{\color{numb}2}}}{1}
      {3}{{{\color{numb}3}}}{1}
      {4}{{{\color{numb}4}}}{1}
      {5}{{{\color{numb}5}}}{1}
      {6}{{{\color{numb}6}}}{1}
      {7}{{{\color{numb}7}}}{1}
      {8}{{{\color{numb}8}}}{1}
      {9}{{{\color{numb}9}}}{1}
      {:}{{{\color{punct}{:}}}}{1}
      {,}{{{\color{punct}{,}}}}{1}
      {\{}{{{\color{delim}{\{}}}}{1}
      {\}}{{{\color{delim}{\}}}}}{1}
      {[}{{{\color{delim}{[}}}}{1}
      {]}{{{\color{delim}{]}}}}{1},
}


% LaTeX Template für Abgaben an der Universität Stuttgart
% Autor: Sandro Speth
% Bei Fragen: Sandro.Speth@studi.informatik.uni-stuttgart.de
%-----------------------------------------------------------
% Modul beinhaltet Befehl fuer Aufgabennummerierung,
% sowie die Header Informationen.

% Überschreibt enumerate Befehl, sodass 1. Ebene Items mit
%\renewcommand{\theenumi}{(\alph{enumi})}
% (a), (b), etc. nummeriert werden.
%\renewcommand{\labelenumi}{\text{\theenumi}}

% Formatierung der Kopfzeile
% \ohead: Setzt rechten Teil der Kopfzeile mit
% Namen und Matrikelnummern aller Bearbeiter
\ohead{Sandro Speth\\
Markus Zilch\\
Dominik Wagner
}
% \chead{} kann mittleren Kopfzeilen Teil sezten
% \ihead: Setzt linken Teil der Kopfzeile mit
% Modulnamen, Semester und Übungsblattnummer
\ihead{Smart Cities and Internet of Things\\
Wintersemester 18/19\\
Specification}

\title{Specification}
\date{Wintersemester 18/19}
\author{Sandro Speth\\
Markus Zilch\\
Dominik Wagner}

% Beginn des eigentlichen Dokuments
\begin{document}
\maketitle

\section*{A1}
The difference between kW and kWh:\\
W is a messuring scale for energy applied per timeinstance.
There are different possibilities to describe W in common terms.
A pretty graphic one is the movement of mass.
1 W equals 1 kg of mass moved by 1 meter in one second: $1 (kg*m*m)/s*s*s$.
Or in electrical terms: 1 W equals 1 Ampere of electrical power with a voltage of 1 Volt.
Both of those formulas are equal to a much simpler Term for Watt: 1 W = 1 J/s.
In simple terms, 1 Watt is the same as one Joule of energy applied over 1 second.
For completeness, 1 kW = 1000 W.\\
\\
Wh are the common term for messuring energyconsumption/-production.
1 Wh is 1 W applied continuously over 1 hour.
 1 Wh = 1 W * 1h = 1 J/s * 3600s = 3600J.
 For a scientific context the Wh therefore is simply not used, instead the common SI standard J is used.\\
\\
In comparison, Wh is the total amount of energy used. W is how much energy is used in a specified timeslot (mostly 1 second).\\
\\
Sources:\\
Robert A. Nelson: The International System of Units. Applied Technology Institute\\
$https://www.aticourses.com/international_system_units.htm$\\
Gérard Borvon: History of the electrical units. S-eau-S, 10. September 2012\\
$http://seaus.free.fr/spip.php?article964$\\
Das Internationale Einheitensystem (SI). Deutsche Übersetzung der BIPM-Broschüre „Le Système international d’unités/The International System of Units (8e édition, 2006)“. In: PTB-Mitteilungen. Band 117, Nr. 2, 2007\\
$https://www.ptb.de/cms/fileadmin/internet/Themenrundgaenge/ImWeltweitenNetzDerMetrologie/si.pdf$\\
aufgrund der EU-Richtlinie 80/181/EWG in den Staaten der EU bzw. dem Bundesgesetz über das Messwesen in der Schweiz\\
$https://www.admin.ch/opc/de/classified-compilation/20101915/$

\section*{A2}

\section*{A3}
Userstories instead of requirements.

\section*{A4}
System architecture Diagram

\section*{A5}
add weather component

\section*{A6}
reliable and responsive system

\section*{A7}
three-tier system architecture

% Ende des Dokuments
\end{document}
