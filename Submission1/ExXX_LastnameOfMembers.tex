% LaTeX Template für Abgaben an der Universität Stuttgart
% Autor: Sandro Speth
% Bei Fragen: Sandro.Speth@studi.informatik.uni-stuttgart.de
%-----------------------------------------------------------
% Hauptmodul des Templates: Hier können andere Dateien eingebunden werden
% oder Inhalte direkt rein geschrieben werden.
% Kompiliere dieses Modul um eine PDF zu erzeugen.

% Dokumentenart. Ersetze 12pt, falls die Schriftgröße anzupassen ist.
\documentclass[12pt]{scrartcl}
% LaTeX Template für Abgaben an der Universität Stuttgart
% Autor: Sandro Speth
% Bei Fragen: Sandro.Speth@studi.informatik.uni-stuttgart.de
%-----------------------------------------------------------
% Modul fuer verwendete Pakete.
% Neue Pakete einfach einfuegen mit dem \usepackage Befehl:
% \usepackage[options]{packagename}
\usepackage[utf8]{inputenc}
\usepackage[T1]{fontenc}
\usepackage[english]{babel}
\usepackage[backend=bibtex]{biblatex}
\usepackage{graphicx}
\usepackage[pdftex,hyperref,dvipsnames]{xcolor}
\usepackage{listings}
\usepackage[a4paper,lmargin={2cm},rmargin={2cm},tmargin={3.5cm},bmargin = {2.5cm},headheight = {4cm}]{geometry}
\usepackage{amsmath,amssymb,amstext,amsthm}
\usepackage[lined,algonl,boxed]{algorithm2e}
\usepackage{tikz}
\usepackage{hyperref}
\usepackage{url}
\usepackage[inline]{enumitem} % Ermöglicht ändern der enum Item Zahlen
\usepackage[headsepline]{scrpage2}
\usepackage{listings} 
\usepackage[nameinlink,noabbrev]{cleveref}
\pagestyle{scrheadings} 
\usetikzlibrary{automata,positioning}
\hypersetup{hidelinks,
	colorlinks=true,
	allcolors=black,
	pdfstartview=Fit,
	breaklinks=true}
	
\colorlet{punct}{red!60!black}
\definecolor{background}{HTML}{EEEEEE}
\definecolor{delim}{RGB}{20,105,176}
\colorlet{numb}{magenta!60!black}

\lstdefinelanguage{json}{
		tabsize=4,
    basicstyle=\normalfont\ttfamily,
    numbers=left,
    numberstyle=\scriptsize,
    stepnumber=1,
    numbersep=8pt,
    showstringspaces=false,
    breaklines=true,
    literate=
     *{0}{{{\color{numb}0}}}{1}
      {1}{{{\color{numb}1}}}{1}
      {2}{{{\color{numb}2}}}{1}
      {3}{{{\color{numb}3}}}{1}
      {4}{{{\color{numb}4}}}{1}
      {5}{{{\color{numb}5}}}{1}
      {6}{{{\color{numb}6}}}{1}
      {7}{{{\color{numb}7}}}{1}
      {8}{{{\color{numb}8}}}{1}
      {9}{{{\color{numb}9}}}{1}
      {:}{{{\color{punct}{:}}}}{1}
      {,}{{{\color{punct}{,}}}}{1}
      {\{}{{{\color{delim}{\{}}}}{1}
      {\}}{{{\color{delim}{\}}}}}{1}
      {[}{{{\color{delim}{[}}}}{1}
      {]}{{{\color{delim}{]}}}}{1},
}


% LaTeX Template für Abgaben an der Universität Stuttgart
% Autor: Sandro Speth
% Bei Fragen: Sandro.Speth@studi.informatik.uni-stuttgart.de
%-----------------------------------------------------------
% Modul beinhaltet Befehl fuer Aufgabennummerierung,
% sowie die Header Informationen.

% Überschreibt enumerate Befehl, sodass 1. Ebene Items mit
%\renewcommand{\theenumi}{(\alph{enumi})}
% (a), (b), etc. nummeriert werden.
%\renewcommand{\labelenumi}{\text{\theenumi}}

% Formatierung der Kopfzeile
% \ohead: Setzt rechten Teil der Kopfzeile mit
% Namen und Matrikelnummern aller Bearbeiter
\ohead{Sandro Speth\\
Markus Zilch\\
Dominik Wagner
}
% \chead{} kann mittleren Kopfzeilen Teil sezten
% \ihead: Setzt linken Teil der Kopfzeile mit
% Modulnamen, Semester und Übungsblattnummer
\ihead{Smart Energy Systems\\
Wintersemester 18/19\\
Report}

\title{Specification}
\date{Wintersemester 18/19}
\author{Sandro Speth\\
Markus Zilch\\
Dominik Wagner}

% Beginn des eigentlichen Dokuments
\begin{document}
\maketitle
\section{Introduction}

The traditional power grid is changing more and more over time.
Due to increasing sensitization for the use of renewable and reliable sources of energy instead of nuclear power sources or fossil energy sources, there is an increasing accommodation of renewable energy.
To fulfill our daily energy need only with such energy sources is quite difficult and needs lots of planning and simulation.
In this work we build a smart energy system to simulate a smart grid.

A smart grid is an energy efficient system with information and communication technology, automation and awareness of energy consumption and supply.
There are many different actors and technologies which are connected to each other and interoperate to optimize the grid.
%TODO Add "`Introduction to smart energy systems"' smart grid slides definitions

A smart energy system creates a bridge between a power grid and a resilient and reliable smart grid.
Users can simulate different energy sources, as well as different kinds of energy consumers, e.g. homes or offices.
Simulation of distributed energy sources and automation of processes build an energy management system with interaction on different levels of the power grid and the consumer grids.
In a smart grid a new aspect is becoming important.
Contrary to currently used normal power grids the exchange of information becomes more and more important.
Especially the consumer side (demand side) of these grids are now being integrated into the power systems in order to ensure optimal power usage and a better understanding of customer needs.
Those new smart grids use various methods to control the flow of energy in order to optimize efficiency and grid usage.
For a fully functional smart grid, the grid has to be made more resistant to any attacks and have mechanisms to self-diagnose problems and take appropriate recovery actions.
The new smart grids offer better ways to integrate renewable but unstable power suppliers like  windturbines and solarpanels through better controllable interactions on a much smaller scale than conventional power grids.
Through these microgrids we can possibly rely completely on renewable energy sources in the future.
If renewable energy sources produce less energy than is needed, the smart grid can buy energy from the main grid.
This process of matching demand and supply can be simulated with our smart energy system.
Using day-ahead energy prices and supply and demand forecasts, the system can simulate demand shifting to be less independent from the main grid.
Therefore our smart energy system can optimize supply, demand, battery data in regard to minimal energy costs with demand shifting and charging/discharging batteries.


The report is structured as follows.
In section \ref{sec:systemdesign} we present our system design.
First, we give some basic foundations which are relevant for our smart energy system, such as the difference between $kWh$ and $kW$.
Afterwards we present our functional requirements for the smart energy systems as user stories.
From these functional requirements we created our architecture which will be described in this section as well. The next section contains the theoretical background to understand the implementation and the description of the basic part of our system. 
In \cref{sec:DemandResponse} we present the mathematical equations the optimizer uses to balance supply and demand in regard to minimal energy costs.
The implementation of the solver is also described in this section.
\Cref{sec:DynamicPricing } shortly describes the day-ahead energy market and our service to collect those prices for the dynamic pricing in our system.
The frontend implementation is described in \cref{sec:frontend}.
In addition to this, we explain which charts we chose to present the data most conveniently to the user.
Finally in \cref{sec:conclusionAndOutlook}, we give a conclusion about the project and present an outlook about possible future work. 

\section{System Design}\label{sec:systemdesign}

\subsection{Difference between $kW$ and $kWh$}\label{sec:diffEnergy}
W is a messuring scale for energy applied per timeinstance.
There are different possibilities to describe $W$ in common terms.
A pretty graphic one is the movement of mass.
$1W$ equals $1kg$ of mass moved by 1 meter in one second: $1 \frac{kg*m^2}{s^3}$.
Or in electrical terms: $1W$ equals 1 Ampere of electrical power with a voltage of 1 Volt.
Both of those formulas are equal to a much simpler Term for Watt: $1 W = 1 J/s$.
In simple terms, 1 Watt is the same as one Joule of energy applied over 1 second.
For completeness, $1kW = 1000 W$.\\
\\
Wh are the common term for messuring energyconsumption/-production.
$1Wh$ is $1W$ applied continuously over 1 hour.
 $1Wh = 1 W * 1h = 1 J/s * 3600s = 3600J$.
 For a scientific context the $Wh$ therefore is simply not used, instead the common SI standard $J$ is used.\\
\\
In comparison, $Wh$ is the total amount of energy used. $W$ is how much energy is used in a specified timeslot (mostly 1 second).\\
\\
Sources:\\
Robert A. Nelson: The International System of Units. Applied Technology Institute\\
\url{https://www.aticourses.com/international_system_units.htm}\\
Gérard Borvon: History of the electrical units. S-eau-S, 10. September 2012\\
\url{http://seaus.free.fr/spip.php?article964}\\
Das Internationale Einheitensystem (SI). Deutsche Übersetzung der BIPM-Broschüre „Le Système international d’unités/The International System of Units (8e édition, 2006)“. In: PTB-Mitteilungen. Band 117, Nr. 2, 2007\\
\url{https://www.ptb.de/cms/fileadmin/internet/Themenrundgaenge/ImWeltweitenNetzDerMetrologie/si.pdf}\\
aufgrund der EU-Richtlinie 80/181/EWG in den Staaten der EU bzw. dem Bundesgesetz über das Messwesen in der Schweiz\\
\url{https://www.admin.ch/opc/de/classified-compilation/20101915/}

\subsection{Difference between consumption and demand}\label{sec:diffconsumptiondemand}
Electricity consumption and electricity demand are two different properties and measured with different measurement units.
The following section contains a description of both and an example at the end.

\subsubsection{Demand}

The demand is the rate of consumption of electricity or mathematical speaking the demand is the derivation of the consumption \cite{StonyBrookUniversity}.
Most of the time the demand is measured in Watt. If you turn on a 100W light bulb, it will demand 100W while it is turned on. At the same time the grid must provide electricity at a rate of 100W. In most cases it is possible to calculate the demand with the following formula \cite{Eggenberger}.
\begin{equation*}
	Demand = Voltage * Current
\end{equation*}
Some customers also have to pay for the demand or peak demand they have because if you have a higher (peak) demand the grid has to support this \cite{StonyBrookUniversity,enertiv}.  %https://www.enertiv.com/resources/what-is-peak-demand,https://www.stonybrook.edu/commcms/energy/facts/demand


\subsubsection{Consumption}

It is easier to understand electricity consumption because we are more used to this concept \cite{StonyBrookUniversity}. Many people deal with electricity consumption while paying their electricity bill because most German electricity meter measure only the consumption. %\cite{https://www.stonybrook.edu/commcms/energy/facts/demand}.
The consumption is the amount of electricity used per time unit \cite{StonyBrookUniversity}\cite{enertiv}. Most of the time the consumption is measured in kilowatt per hour.
The formula to calculate the consumption is the following \cite{StonyBrookUniversity}. %https://www.stonybrook.edu/commcms/energy/facts/demand.

\begin{equation*}
	Consumption = Demand * Time
\end{equation*}
For example, 5 W LED bulbs turned on for 1h have the consumption of 5 Wh.

\subsubsection{Difference}

\begin{figure}[!h]
	\centering
	\includegraphics[width=1.0\textwidth]{../figures/DemandConsumption.pdf}
	\caption{Example for demand and consumption}
	\label{fig:demandConsumption}
\end{figure}

The demand is the rate of which we use energy and the consumption is the total energy used for a given time frame \cite{StonyBrookUniversity}. The formulas also show this relation between the consumption and the demand. The consumption is the demand multiplied with the time. If someone turns on 1 heating unit with a demand of 1kW for 2 hours than the demand during this hours is 1kW, but the consumption is 2kWh. The consumption is the same if two heating units are used for half an hour, but the demand is doubled (2kW). Figure \ref{fig:demandConsumption} contains another but similar example. To put it in simple terms the demand is comparable with the speed of a car and the consumption is the distance you drive. The faster you drive the more distance is accumulated over time.


\subsection{Userstories}\label{sec:userstories}
\subsection{User Story Template}
Template
As a <type of user>, I want/need <some goal> so that <some reason>

\subsection{User Stories}
\begin{enumerate}
\item As a user, I need to create a wind turbine in the simulation so that I can calculate the potential energy output.

\item As a user, I need to create a photovoltaic panel in the simulation so that I can calculate the potential energy output.

\item As a user, I need to create a battery in the simulation so that I can \textbf{visualize} the impact of the energy storage on the grid.

\item As a user, I need to create a home on the demand side in the simulation so that I can calculate the energy demand of the home.

\item As a user, I need to create a commercial building on the demand side in the simulation so that I can calculate the energy demand of the commercial building.

\item As a user, I need to  get continuously calculate dynamic prices per kWh form the simulation so that I can determine if I want to sell my produced energy or store it for later use or selling.


\item As a user, I need to use wetter data in the simulation so that the simulation gets more precise.

\item As a user, I need to be able to use already saved wetter data in the simulation so that the dependency on the availability of the wetter service is reduced.

\item As a user, I need to use the simulation to generate a forecast for energy generation and demand so that I can make informed decisions.

\item As a user, I want to add customizable suppler modules to the simulation so that ...

\item As a user, I want to add customizable consumer modules to the simulation so that ...

\item As a user, I need to run the simulation in an island mode which does not contain a connection to the main power grid so that I'm independent of the main power grid and its failures.

\item As a user, I need to get a visual notification if the supply is smaller than the demand so that I know that I have to add more energy supplier.

\item As a user, I need that the battery is used if the supply is smaller than the demand and it has enough charge so that ...

\item As a user, I need that the battery is recharged if the demand is greater than the supply so that ...

\item As a user, I need that the battery is recharged if the demand is greater than the supply so that ...

\item As a user, I need that the main power is used to manage peak demands if the microgrid has a connection to the main power so that ...

\item As a user, I need that the demand side simulation features standard loads like lighting so that...

\item As a user, I need that the demand side consumers feature different load scenarios like home users and commercial users (constant load, occasionally peak loads) so that ...

\item As a user, I need to use the system with my webbrowser so that I can use different platforms to view it.

\item As a user, I need to visualize different metrics so that ...

\item As a user, I need to view the current electric demand of the grid so that...

\item As a user, I need to view the electric consumption of the grid so that...

\item As a user, I want to adjust the demand by postponing the use of devices during peak hours so that...

\item As a user, I need that the simulation adapts the demand based the price per kWh so that...



\end{enumerate}
Todo:\\
\begin{enumerate}
	\item Replace ... with content
	\item check spelling and grammar
	\item price per min and price of the next day
	\item user stories for visualization
\end{enumerate}




\subsection{Architecture}
System architecture Diagram

\subsection*{A5}
add weather component

\subsection*{A6}
reliable and responsive system

\subsection*{A7}
three-tier system architecture

% Ende des Dokuments
\end{document}
