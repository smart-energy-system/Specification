\subsection{Balance within the microgrid}
To guarantee the balance in the microgrid as defined in the slides we have to formulate a mathematical equation to use as a constraint.
We came up with the following equation:\\
\begin{align} \label{eq1:balance}
\begin{split}
\forall t \in T: \sum\limits_{s \in S}{P_{s}(t)} + \sum\limits_{b \in B}{P_{b}(t)} + P_{g}(t) = \sum_{h \in H}({P_{h}(t)} + PposShift_h(t) - PnegShift_h(t))
\\ + \sum_{c \in C}({P_{c}(t)} + PposShift_c(t) - PnegShift_c(t)) - \sum\limits_{b \in B}{Pd_{b}(t)}
\end{split}
\end{align}
As one can see, for all timesteps $t$ in the set of all timesteps $T$ the equation has to hold true. 
This is the constraint for the whole microgrid. 
The left side of the equation represents the power supplied by the supplier ($P_{s}(t)$), the batteries ($P_{b}(t)$) and the import or export from the main grid ($P_{g}(t)$), the right side represents the consumers, which contains homes ($P_{h}(t)$) and commercial buildings ($P_{c}(t)$), the batteries ($Pd_{b}(t)$) while charging. 
On the right side, there are also the demand shift values for homes and commercial buildings.

If the balance between the consumers and the suppliers is disturbed, severe consequences can happen. 
Any imbalance results in a change in the frequency. 
This change may forces more suppliers to stop production because most suppliers can only operate in a narrow frequency range. 
As a result of this the imbalance ingresses even more. 
If the change in the frequency exceeds a certain threshold suppliers and consumers still connected to the grid could be damaged. 

The supply side of the equation itself is compartmentalized into different factors which are explained in the following.

\textbf{Wind turbines and solar panels:} On the supply side we have the sum the power in timestep $t$ of all suppliers of the system (($P_{s}(t)$)), regardless of wind turbines or solar panels.
We get those values from our model given data for the timestep and weather conditions. 
Our software calculates demand forecast and supply forecast based on models of the supplier and consumers. 
Additionally, the location of the installations are provided by the user and a weather forecast is provided by a weather service. 
For example, for the wind turbines, the Herman Wobus equation is used to calculate the saturated vapor pressure \cite{NOAA}. 
This equation is based on empirical data and helps us to incorporate the weather forecast in the wind supply forecast. 
Because all these calculations happen before the optimization step, the solver does not need to know anything about the specifics of the supplier or consumer. As a result of this we don`t need any constrains for the consumer or suppliers. 
This supports reuse-ability of the solver and the models for it. 
The forecast data is provided as an array of data which contains a value for each timestep. 
The values represent the supply or demand and are measured in $Watts$.  %TODO Auf welche Einheit haben wir uns jetzt entschieden?

\textbf{Battery supply:} The second part is the power provided by the sum of all batteries at a given timestep $t$. Additionally, there is a limit on the charge and the discharge rate. Furthermore, charging is not 100\% efficient. Each battery has a fill level for each time step. It holds the information how much charge is stored in the battery at each step. This results in the following constraints.
\begin{align} \label{eq:limitDischarge}
\begin{split}
\forall b \in B, \forall t \in T: P_{b}(t) < maxDischargeRate_{b}
\end{split}
\end{align}
\begin{align} \label{eq:limitCharge}
\begin{split}
\forall b \in B, \forall t \in T: Pd_{b}(t) < maxChargeRate_{b}
\end{split}
\end{align}
\begin{align} \label{eq:effi}
\begin{split}
\forall b \in B, \forall t \in T: fillLevel_{b}(t+1) = \cfrac{Pd_{b}(t)*chargeEfficiency_b}{100}
\end{split}
\end{align}
 % TODO wiederholt sich das nicht von oben?

\textbf{Import and export to the main grid:} The connection to the main grid is resolved in the third part of the supply side of the equation.
If we have more demand than supply, the solver will either use the energy stored in the battery or import energy from the main grid. While the demand is satisfied by the suppliers and the battery, no further energy import is necessary. But if the batteries are empty and the demand exceeds the supply more energy has to be imported from the main grid. The import of energy form the main grid costs money. If the supply exceeds the demand it is possible to charge the batteries or to export the energy to the main grid. Exporting energy to the main grid provides revenue.

For the demand side, we chose a similar approach.

\textbf{Demand from homes and commercial buildings:} The first one is the sum over all homes of a a given timestep $t$ with related demand shift of every single home to calculate the total demand of all homes.
The second one is the sum over all commecial buildings of a given timestep $t$ with related demand shift of every single commecial building to calculate the total demand off all commecial buildings.
Comparable to the supply forecast, the demand forecast is generated before the optimization step. 
It is based on the demand profiles described in \cref{subsec:Demand}. 
The demand of the homes is provided as an array of constants to the solver and measured in $Watts$. 
Because the supply and the demand are provided as constants to the optimization component, we did not formulate any constraints for the supplier or the consumers.
For the demand shift parts of the demand entities we have a positive shift and a negativ shift.
The negative demand shift tells us how much demand is shifted away on this timestep.
The positive demand shift tells us the opposite, how much demand is shifted to this timestep.
So the negative demand shift will decrease the demand while the positive demand shift will increase the demand.
If the negative demand shift of a home or commecial building is greater than 0, the positive demand shift has to stay 0 and vice versa.

\textbf{Demand of batteries:}
The third part calculates the demand of all batteries at a given timestep $t$.
If we have more supply than demand, the solver can either export the unnecessarily produced energy to the main grid or store it in a battery for later use.
The storage of energy in the battery has a small loss of energy.

\subsubsection{Additional Constraints}
There are additional constrains which have to hold true in our simulation.

We don't want the batteries charge and discharge at the same time.
So we need a constraint to prohibit this kind of behaviour as shown in \Cref{eq:batteryConstraint}
\begin{align} \label{eq:batteryConstraint}
\begin{split}
battery_{supply} = \sum\limits_{b\in B} P_{b}(t)\\
battery_{demand} = \sum\limits_{b\in B} Pd_{b}(t)\\
(battery_{supply} = 0 \wedge battery_{demand} = 0 ) \vee (battery_{supply} > 0 \veebar battery_{demand} > 0)
\end{split}
\end{align}

There is also such an constraint for the shifting part to prohibit positive and negative shifting at the same time.

We need to make an upper bound for the positive shifting which a user can provide for the house or commecial building demand, e.g. 30\%.
The constraint for the house shifting is shown in \Cref{eq:shiftingUpperBound}.
The constraint for the commecial building is the same.
\begin{align} \label{eq:shiftingUpperBound}
\begin{split}
PposShift_h(t) = \cfrac{P_{h}(t) * n}{100}
\end{split}
\end{align}

There are two possiblilities for shifting demand which can be chosen by a user.
The first one is used if the optimizer is not strong enough to solve variable shifting for every timestep.
In this version, we allow the optimizer to shift one time for a home with the constraints shown in \Cref{eq:numberOfShifts}.
We did the same for the commecial buildings.
The second version allows variable shifting for every timestep.
\begin{align} \label{eq:numberOfShifts}
\begin{split}
\forall h \in H: sumOfPosShifts_{h} = \sum\limits_{t = 0}^{t_{max}} PposShift_h(t)\\
sumOfPosShifts_{h} \leq 1
\end{split}
\end{align}

The sum of negative shifting is set equals the sum of positiv shifting to build the balance between them.
In addition to this, the $PnegShift_h(t)$ has to be less or equals than 1 too.
So every building can shift at maximum once  n\% of their demand to another timestep.

\subsection{Objective of the optimization problem}
We import or export energy from the main grid with $P_{g}(t)$.
If $P_{g}(t) < 0$ than energy will be exported since the overall supply gets less than only with the batteries and renewable supplier.
If $P_{g}(t) > 0$ than energy will be imported since the overall supply gets more than only with the batteries and renewable supplier.
We can use this to compute our profit or loss since we want the optimizer to maximize the profit.
To compute profit and loss we declared two variables $isEnergyImport$ and $isEnergyExport$ as shown in \Cref{eq:importexport}.
\begin{align} \label{eq:importexport}
\begin{split}
P_{g}(t) < 0 \Rightarrow isEnergyImport\\
P_{g}(t) > 0 \Rightarrow isEnergyExport
\end{split}
\end{align}
If $isEnergyImport$ is true, the optimizer will use the absolut value of $P_{g}$ as import.
In this case the export will be zero for timestep $t$.
If $isEnergyExport$ is true, it's the other way round.


To maximize the profit in a specified interval of timesteps $t \in T$ between the timesteps $a \in T$ and $b \in T$ we propose the following function:
\begin{equation} \label{eq:opt}
\max_{p \in P}{(\sum_{t=a}^{b}{((E(t)*price(t))-(I(t)*cost(t)))})}
\end{equation}
$P$ is th set of all possible decision configurations.
$E(t)$ contains the energy export of timestep $t$ and $I(t)$ contains the energy import of timestep $t$.
These variable are depending on the decisions made by the optimizer on the equation for the balance in the microgrid of the exercises before. 
All configurations from $P$ have to satisfy all the constraints. 
The formula consists of two main parts. 
The minuend in the sum calculates the profit for exporting energy to the main grid for these timestep. 
For this calculation, the total energy export $E(t)$ for this timestep is multiplied by the price for the energy. 
The subtrahend in the sum expresses the cost for importing energy from the main grid. 
It consists of the total imported energy $I(t)$ multiplied by the cost for importing energy at this timestep. 
There are different possibilities to improve the total profit. 
For example, one could sell energy at a high price. 
Another possibility is to store energy if the price for selling is low. 
For our simulation we use timesteps with the width of one hour, so the values for power and energy are the same, but $E(t)$ and $I(t)$ are measured in $Wh$ while the unit of all values of the previous equation is $Watt$.

\subsection{Nomenclature Table}
	\begin{longtable}{|c|p{.50\textwidth}|c|c|}
		\toprule
		\textbf{Symbol} & \textbf{Description} & \textbf{Unit} & \textbf{Equation} \\ \midrule
		T & The set of all timesteps. The step width is 1h and usually 24 steps are used. & Hour & \Cref{eq1:balance,eq:opt} \\ \midrule
		t & A single step from the set T. & Hour & \Cref{eq1:balance,eq:opt} \\ \midrule
		S & The set of all suppliers (Different wind turbines, solar panels...). & Watt & \Cref{eq1:balance} \\ \midrule
		s & A single supplier from the set S. & & \Cref{eq1:balance} \\ \midrule
		$P_{s}(t)$ & The energy supply from a single supplier for the specific timestep t. & Watt & \Cref{eq1:balance} \\ \midrule
		B & The set of all batteries. & & \Cref{eq1:balance} \\ \midrule
		b & A single battery. & & \Cref{eq1:balance} \\ \midrule
		$P_{s}(t)$ & The energy supply from a single battery for the specific timestep t. & Watt & \Cref{eq1:balance} \\ \midrule
		$P_{g}(t)$ & The energy supply from the main grid for the specific timestep t. & Watt & \Cref{eq1:balance} \\ \midrule
		D & The set of all consumers (Homes, commercial buildings) & & \Cref{eq1:balance} \\ \midrule
		d & A single consumer from the set D & & \Cref{eq1:balance} \\ \midrule
		$P_{d}(t)$ & The energy demand for a single consumer at a specific timestep t. & Watt & \Cref{eq1:balance} \\ \midrule
		$P_{b}(t)$ & The energy demand for a single battery at a specific timestep t. & Watt & \Cref{eq1:balance} \\ \midrule
		$Pd_{g}(t)$ & The energy exported by the microgrid to the main grid at a specific timestep t. & Watt & \Cref{eq1:balance} \\
		P & The set of all possible configurations &  & \Cref{eq:opt} \\ \midrule
		p & A single configuration &  & \Cref{eq:opt} \\ \midrule
		a & A single step form the set T. It marks the first timestep of the optimization & Hour & \Cref{eq:opt} \\ \midrule
		b & A single step form the set T. It marks the last timestep of the optimization & Hour & \Cref{eq:opt} \\ \midrule
		E(t) & The total energy exported to the main grid for a specific timestep t. & Wh & \Cref{eq:opt} \\ \midrule
		I(t) & The total energy imported from the main grid for a specific timestep t. & Wh & \Cref{eq:opt} \\ \midrule
		price(t) & The function provides the price for energy per Wh which is sold to the main grid for a specific timestep t. & Cent per Wh & \Cref{eq:opt} \\ \midrule
		cost(t) & The function provides the cost for energy per Wh which is imported from the main grid for a specific timestep t. & Cent per Wh& \Cref{eq:opt} \\
		\bottomrule
			\caption[Nomenclature Table]{Describes every Symbol}
		\label{tab:Ergebnisse}
	\end{longtable}


