\subsection{Balance within the microgrid}
To guarantee the balance in the microgrid as defined in the slides we have to formulate a mathematical equation to use as a constraint.
We came up with the following equation:\\
$\forall t \in T: \sum_{s \in S}{s_{prod}(t)} + \sum_{b \in B}{b_{prod}(t)} + G_{prod}(t) = \sum_{d \in D}{d_{demand}(t)} = \sum_{b \in B}{b_{demand}(t)} + G_{demand}(t)$\\
As one can see, for all time steps t in the set of all time steps T the equation has to hold true.
This is the constraint for the whole microgrid.
The equation itself is compartmentalized into different factors.
On the supply side we have the sum the produced energy in time step t of all suppliers of the system, regardless of wind or solar energy.
We get those values from our model given data for the time step and weather conditions.
The second part is the energy provided by all batteries at a given time step.
The connection to the main grid is resolved in the third part of the supply side of the equation.
It gives the energy provided by the main grid at a given time step t.\\
For the demand side we chose a similar  approach.
We split the equation into several smaller parts.
The first one is the sum over all energy demand of all  demander at a given time step.
The second one is the again the part for the batteries.
Here we calculate the energy demand of all batteries at given time step.
In the last part we again have the connection to the main grid.
Here we add the energy we provide to the main grid, in other words what the main grid can demand, at a given time step.\\<