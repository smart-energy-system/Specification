% LaTeX Template für Abgaben an der Universität Stuttgart
% Autor: Sandro Speth
% Bei Fragen: Sandro.Speth@studi.informatik.uni-stuttgart.de
%-----------------------------------------------------------
% Hauptmodul des Templates: Hier können andere Dateien eingebunden werden
% oder Inhalte direkt rein geschrieben werden.
% Kompiliere dieses Modul um eine PDF zu erzeugen.
% Dokumentenart. Ersetze 12pt, falls die Schriftgröße anzupassen ist.
\documentclass[12pt]{scrartcl}
% LaTeX Template für Abgaben an der Universität Stuttgart
% Autor: Sandro Speth
% Bei Fragen: Sandro.Speth@studi.informatik.uni-stuttgart.de
%-----------------------------------------------------------
% Modul fuer verwendete Pakete.
% Neue Pakete einfach einfuegen mit dem \usepackage Befehl:
% \usepackage[options]{packagename}
\usepackage[utf8]{inputenc}
\usepackage[T1]{fontenc}
\usepackage[english]{babel}
\usepackage[backend=bibtex]{biblatex}
\usepackage{graphicx}
\usepackage[pdftex,hyperref,dvipsnames]{xcolor}
\usepackage{listings}
\usepackage[a4paper,lmargin={2cm},rmargin={2cm},tmargin={3.5cm},bmargin = {2.5cm},headheight = {4cm}]{geometry}
\usepackage{amsmath,amssymb,amstext,amsthm}
\usepackage[lined,algonl,boxed]{algorithm2e}
\usepackage{tikz}
\usepackage{hyperref}
\usepackage{url}
\usepackage[inline]{enumitem} % Ermöglicht ändern der enum Item Zahlen
\usepackage[headsepline]{scrpage2}
\usepackage{listings} 
\usepackage[nameinlink,noabbrev]{cleveref}
\pagestyle{scrheadings} 
\usetikzlibrary{automata,positioning}
\hypersetup{hidelinks,
	colorlinks=true,
	allcolors=black,
	pdfstartview=Fit,
	breaklinks=true}
	
\colorlet{punct}{red!60!black}
\definecolor{background}{HTML}{EEEEEE}
\definecolor{delim}{RGB}{20,105,176}
\colorlet{numb}{magenta!60!black}

\lstdefinelanguage{json}{
		tabsize=4,
    basicstyle=\normalfont\ttfamily,
    numbers=left,
    numberstyle=\scriptsize,
    stepnumber=1,
    numbersep=8pt,
    showstringspaces=false,
    breaklines=true,
    literate=
     *{0}{{{\color{numb}0}}}{1}
      {1}{{{\color{numb}1}}}{1}
      {2}{{{\color{numb}2}}}{1}
      {3}{{{\color{numb}3}}}{1}
      {4}{{{\color{numb}4}}}{1}
      {5}{{{\color{numb}5}}}{1}
      {6}{{{\color{numb}6}}}{1}
      {7}{{{\color{numb}7}}}{1}
      {8}{{{\color{numb}8}}}{1}
      {9}{{{\color{numb}9}}}{1}
      {:}{{{\color{punct}{:}}}}{1}
      {,}{{{\color{punct}{,}}}}{1}
      {\{}{{{\color{delim}{\{}}}}{1}
      {\}}{{{\color{delim}{\}}}}}{1}
      {[}{{{\color{delim}{[}}}}{1}
      {]}{{{\color{delim}{]}}}}{1},
}


% LaTeX Template für Abgaben an der Universität Stuttgart
% Autor: Sandro Speth
% Bei Fragen: Sandro.Speth@studi.informatik.uni-stuttgart.de
%-----------------------------------------------------------
% Modul beinhaltet Befehl fuer Aufgabennummerierung,
% sowie die Header Informationen.

% Überschreibt enumerate Befehl, sodass 1. Ebene Items mit
%\renewcommand{\theenumi}{(\alph{enumi})}
% (a), (b), etc. nummeriert werden.
%\renewcommand{\labelenumi}{\text{\theenumi}}

% Formatierung der Kopfzeile
% \ohead: Setzt rechten Teil der Kopfzeile mit
% Namen und Matrikelnummern aller Bearbeiter
\ohead{Sandro Speth\\
Markus Zilch\\
Dominik Wagner
}
% \chead{} kann mittleren Kopfzeilen Teil sezten
% \ihead: Setzt linken Teil der Kopfzeile mit
% Modulnamen, Semester und Übungsblattnummer
\ihead{Smart Energy Systems\\
Wintersemester 18/19\\
Report}
\bibliography{bibliography.bib}
\title{Practical course
\\ Smart Energy Systems\\
	Report}
\date{Wintersemester 18/19}
\author{Sandro Speth\\
Markus Zilch\\
Dominik Wagner}

% Beginn des eigentlichen Dokuments
\begin{document}
\maketitle
\section{Introduction}

The traditional power grid is changing more and more over time.
Due to increasing sensitization for the use of renewable and reliable sources of energy instead of nuclear power sources or fossil energy sources, there is an increasing accommodation of renewable energy.
To fulfill our daily energy need only with such energy sources is quite difficult and needs lots of planning and simulation.
In this work we build a smart energy system to simulate a smart grid.

A smart grid is an energy efficient system with information and communication technology, automation and awareness of energy consumption and supply.
There are many different actors and technologies which are connected to each other and interoperate to optimize the grid.
%TODO Add "`Introduction to smart energy systems"' smart grid slides definitions

A smart energy system creates a bridge between a power grid and a resilient and reliable smart grid.
Users can simulate different energy sources, as well as different kinds of energy consumers, e.g. homes or offices.
Simulation of distributed energy sources and automation of processes build an energy management system with interaction on different levels of the power grid and the consumer grids.
In a smart grid a new aspect is becoming important.
Contrary to currently used normal power grids the exchange of information becomes more and more important.
Especially the consumer side (demand side) of these grids are now being integrated into the power systems in order to ensure optimal power usage and a better understanding of customer needs.
Those new smart grids use various methods to control the flow of energy in order to optimize efficiency and grid usage.
For a fully functional smart grid, the grid has to be made more resistant to any attacks and have mechanisms to self-diagnose problems and take appropriate recovery actions.
The new smart grids offer better ways to integrate renewable but unstable power suppliers like  windturbines and solarpanels through better controllable interactions on a much smaller scale than conventional power grids.
Through these microgrids we can possibly rely completely on renewable energy sources in the future.
If renewable energy sources produce less energy than is needed, the smart grid can buy energy from the main grid.
This process of matching demand and supply can be simulated with our smart energy system.
Using day-ahead energy prices and supply and demand forecasts, the system can simulate demand shifting to be less independent from the main grid.
Therefore our smart energy system can optimize supply, demand, battery data in regard to minimal energy costs with demand shifting and charging/discharging batteries.


The report is structured as follows.
In section \ref{sec:systemdesign} we present our system design.
First, we give some basic foundations which are relevant for our smart energy system, such as the difference between $kWh$ and $kW$.
Afterwards we present our functional requirements for the smart energy systems as user stories.
From these functional requirements we created our architecture which will be described in this section as well. The next section contains the theoretical background to understand the implementation and the description of the basic part of our system. 
In \cref{sec:DemandResponse} we present the mathematical equations the optimizer uses to balance supply and demand in regard to minimal energy costs.
The implementation of the solver is also described in this section.
\Cref{sec:DynamicPricing } shortly describes the day-ahead energy market and our service to collect those prices for the dynamic pricing in our system.
The frontend implementation is described in \cref{sec:frontend}.
In addition to this, we explain which charts we chose to present the data most conveniently to the user.
Finally in \cref{sec:conclusionAndOutlook}, we give a conclusion about the project and present an outlook about possible future work. 

\section{System Design}\label{sec:systemdesign}

\subsection{Difference between $kW$ and $kWh$}\label{sec:diffEnergy}
W is a measuring scale for energy applied per time instance.
There are different possibilities to describe $W$ in common terms.
A pretty graphic one is the movement of mass.
$1W$ equals $1kg$ of mass moved by 1 meter in one second: $1 \frac{kg*m^2}{s^3}$.
Or in electrical terms: $1W$ equals 1 Ampere of electrical power with a voltage of 1 Volt.
Both of those formulas are equal to a much simpler Term for Watt: $1 W = 1 J/s$.
In simple terms, 1 Watt is the same as one Joule of energy applied over 1 second.
For completeness, $1kW = 1000 W$ \cite{Nelson,Borvon,SIStandard}.

$Wh$ are the common term for measuring energy consumption and production.
$1Wh$ is $1W$ applied continuously over 1 hour.
$1Wh = 1 W * 1h = 1 J/s * 3600s = 3600J$.
For a scientific context the $Wh$ therefore is simply not used, instead the common SI standard $J$ is used.
In comparison, $Wh$ is the total amount of energy used. $W$ is how much energy is used in a specified timeslot (mostly 1 second) \cite{EURichtlinie,Bundesgesetz}.


\subsection{Difference between consumption and demand}\label{sec:diffconsumptiondemand}
Electricity consumption and electricity demand are two different properties and measured with different measurement units.
The following section contains a description of both and an example at the end.

\subsubsection{Demand}

The demand is the rate of consumption of electricity or mathematical speaking the demand is the derivation of the consumption \cite{StonyBrookUniversity}.
Most of the time the demand is measured in Watt. If you turn on a 100W light bulb, it will demand 100W while it is turned on. At the same time the grid must provide electricity at a rate of 100W. In most cases it is possible to calculate the demand with the following formula \cite{Eggenberger}.
\begin{equation*}
	Demand = Voltage * Current
\end{equation*}
Some customers also have to pay for the demand or peak demand they have because if you have a higher (peak) demand the grid has to support this \cite{StonyBrookUniversity,enertiv}.  %https://www.enertiv.com/resources/what-is-peak-demand,https://www.stonybrook.edu/commcms/energy/facts/demand


\subsubsection{Consumption}

It is easier to understand electricity consumption because we are more used to this concept \cite{StonyBrookUniversity}. Many people deal with electricity consumption while paying their electricity bill because most German electricity meter measure only the consumption. %\cite{https://www.stonybrook.edu/commcms/energy/facts/demand}.
The consumption is the amount of electricity used per time unit \cite{StonyBrookUniversity}\cite{enertiv}. Most of the time the consumption is measured in kilowatt per hour.
The formula to calculate the consumption is the following \cite{StonyBrookUniversity}. %https://www.stonybrook.edu/commcms/energy/facts/demand.

\begin{equation*}
	Consumption = Demand * Time
\end{equation*}
For example, 5 W LED bulbs turned on for 1h have the consumption of 5 Wh.

\subsubsection{Difference}

\begin{figure}[!h]
	\centering
	\includegraphics[width=1.0\textwidth]{../figures/DemandConsumption.pdf}
	\caption{Example for demand and consumption}
	\label{fig:demandConsumption}
\end{figure}

The demand is the rate of which we use energy and the consumption is the total energy used for a given time frame \cite{StonyBrookUniversity}. The formulas also show this relation between the consumption and the demand. The consumption is the demand multiplied with the time. If someone turns on 1 heating unit with a demand of 1kW for 2 hours than the demand during this hours is 1kW, but the consumption is 2kWh. The consumption is the same if two heating units are used for half an hour, but the demand is doubled (2kW). Figure \ref{fig:demandConsumption} contains another but similar example. To put it in simple terms the demand is comparable with the speed of a car and the consumption is the distance you drive. The faster you drive the more distance is accumulated over time.


\subsection{Userstories}\label{sec:userstories}
\subsection{User Story Template}
Template
As a <type of user>, I want/need <some goal> so that <some reason>

\subsection{User Stories}
\begin{enumerate}
\item As a user, I need to create a wind turbine in the simulation so that I can calculate the potential energy output.

\item As a user, I need to create a photovoltaic panel in the simulation so that I can calculate the potential energy output.

\item As a user, I need to create a battery in the simulation so that I can \textbf{visualize} the impact of the energy storage on the grid.

\item As a user, I need to create a home on the demand side in the simulation so that I can calculate the energy demand of the home.

\item As a user, I need to create a commercial building on the demand side in the simulation so that I can calculate the energy demand of the commercial building.

\item As a user, I need to  get continuously calculate dynamic prices per kWh form the simulation so that I can determine if I want to sell my produced energy or store it for later use or selling.


\item As a user, I need to use wetter data in the simulation so that the simulation gets more precise.

\item As a user, I need to be able to use already saved wetter data in the simulation so that the dependency on the availability of the wetter service is reduced.

\item As a user, I need to use the simulation to generate a forecast for energy generation and demand so that I can make informed decisions.

\item As a user, I want to add customizable suppler modules to the simulation so that ...

\item As a user, I want to add customizable consumer modules to the simulation so that ...

\item As a user, I need to run the simulation in an island mode which does not contain a connection to the main power grid so that I'm independent of the main power grid and its failures.

\item As a user, I need to get a visual notification if the supply is smaller than the demand so that I know that I have to add more energy supplier.

\item As a user, I need that the battery is used if the supply is smaller than the demand and it has enough charge so that ...

\item As a user, I need that the battery is recharged if the demand is greater than the supply so that ...

\item As a user, I need that the battery is recharged if the demand is greater than the supply so that ...

\item As a user, I need that the main power is used to manage peak demands if the microgrid has a connection to the main power so that ...

\item As a user, I need that the demand side simulation features standard loads like lighting so that...

\item As a user, I need that the demand side consumers feature different load scenarios like home users and commercial users (constant load, occasionally peak loads) so that ...

\item As a user, I need to use the system with my webbrowser so that I can use different platforms to view it.

\item As a user, I need to visualize different metrics so that ...

\item As a user, I need to view the current electric demand of the grid so that...

\item As a user, I need to view the electric consumption of the grid so that...

\item As a user, I want to adjust the demand by postponing the use of devices during peak hours so that...

\item As a user, I need that the simulation adapts the demand based the price per kWh so that...



\end{enumerate}
Todo:\\
\begin{enumerate}
	\item Replace ... with content
	\item check spelling and grammar
	\item price per min and price of the next day
	\item user stories for visualization
\end{enumerate}




\subsection{Architecture}
The weather component is the component responsible for looking up weather data for registered supply and demand components.
The component reads registration data from the database and looks up weather data for each recorded entry.
The collected data is written back into the database in order to be usable for the rest of the system.\\
\\
In order for the whole system to be reliable and responsive we devided the weather component from the rest of the system.
The weather component speaks only to the weather database where it get the locations for which it should collect weather data.
The collected data gets written into the database and stored.
Even in the case of a failure of the weather component the already collected data is still accessible and therefore the system can still access it, making the system independent of the weather component.
In the case of a new registration of a component during a weather component failure, the database may return default values for the new component in a way the running system is not being halted by missing data.
These measures should make the system reliable and and responsive in that context.\\
\\
The three-layered architecture was already planned for in the first conceptions of the project, therefore no extra steps had to be taken.
We divided the project in subcomponents and put them in the respective part of the architecture.\\
The first layer is the data layer which only consists of data-providing hardware and databases. Since the current project does not include data-providing hardware (as far as the current conception goes) only databases are left in this layer.\\
The second layer consists of the functional components of the project.
All suppliers, storages, consumers and utility components are part of this layer, since they mostly take data from the databases and compute their respective power in- and output based on those values and give them to the simulation component.
The simulation component is part of this layer, and takes the data the other component in order to model the different interactions of the smart grid.
The workflow is controlled by the controller component and connects all components together.
The last component in this layer is the REST component which makes the computed data from the simulation component available to the frontend layer.\\
The frontend layer is the third and last layer in our project.
It handles the interaction with the user and makes it possible for the user to add and remove components to the smart grid simulation as he sees fit.

\section{Implementation}\label{sec:Implementation}
\subsection{Geather weather design}

\subsection{Windturbine}
\subsubsection{Swept area function}
The swept area of a windturbine is dependent on the radius of the rotary blades. Combined with the constant Pi the area swept can be computed by the equation: $A = r^{2}* \Pi $
\subsubsection{Vapor pressure}
To compute the actual vapor pressure in the air the relative humidity has to be given. The relative humidity is computable with the equation $H = P_{av} / P_s$ with $P_{av}$ being the actual vapor pressure and $P_s$ being the saturated vapor pressure at a given temperature. This equation can be restructured to get the actual vapor pressure: $P_{av}  = H * P_s$. The relative humidity is a parameter for the function, which is filled with data form the before mentioned weather data collection.\\
\\
To be able to now compute the actual vapor pressure at a given temperature we still need the saturated vapor pressure. For this we can use the Herman Wobus equation (E being the vapor pressure):\\
$E = e/p$ with\\
e = 6.1078 and\\
$p = c_0 + T * (c_1 + T * (c_2 + T * (c_3 + T * (c_4 + T *(c_5 + T * (c_6 + T * (c_7 + T * c_8 + T * c_9)))))))$\\
with $c_0$ to $c_9$ being constants and T being the temperature in degrees Celsius.
\subsubsection{Density of moist air}
In order to compute the density of moist air, we have to have a look at how it is compounded. Moist air density is a mixture of dry air and water vapor. The physical equation for this is: $D_m = (P_d/(R_d * T_k))+(P_v/(R_v*T_k))$\\
$D_m$ is the density of moist air, $P_d$ is the pressure of dry air at the specified temperature, $R_d$ is the gas constant for dry air, $P_v$ is the pressure of water vapor at the specified temperature, $R_v$ is the gas constant for water vapor and $T_k$ is the given temperature in degrees Kelvin. With the previously implemented methods this equation system is able to compute the air density with only the temperature and relative humidity given.
\subsubsection{Wind turbine model}
The wind turbine model contains some variables which can be set by the user. The function for computing the generated energy is derived from the power coefficient of the turbine (basically the efficiency), the size of the turbine (shows in area swept), the density of the air in the area and of course the weather conditions which apply at the moment given.
The general equation for this setup is: $P_{avail} = (1/2) * p * A * v^{3} * C$
where p is the air pressure, A the area swept, v the windspeed and C the power coefficient.\\
Source: https://www.raeng.org.uk/publications/other/23-wind-turbine
\subsection{Photovoltaic}
\subsubsection{Temperature loss}
The function for the temperature loss is giving a linear function for the percentage loss of energy depending on the degrees over 25 degree Celsius. We made the assumption the efficiency does not go over 100 \% even for temperatures below 25 degree Celsius.The resulting equation is as follows: $L_t = min((T-25)*-0.005 , 0)$
\subsubsection{Performance Ratio}
The performance ration is computed with the equation $1.0 - (\text{total percentage of losses})$.
With the previous computed losses for temperature and the constant loss $L_0$ of 0.14 we get the equation $R = 1.0 - (L_0 + L_t) = 1.0 - (0.14 + L_t)$
\subsubsection{Solar irradiance}

\section{Demand Response and coordination}\label{sec:DemandResponse}
\subsection{Balance within the microgrid}
To guarantee the balance in the microgrid as defined in the slides we have to formulate a mathematical equation to use as a constraint.
We came up with the following equation:\\
\begin{equation} \label{eq1}
\forall t \in T: \sum_{s \in S}{s_{prod}(t)} + \sum_{b \in B}{b_{prod}(t)} + G_{prod}(t) = \sum_{d \in D}{d_{demand}(t)} = \sum_{b \in B}{b_{demand}(t)} + G_{demand}(t)
\end{equation}
As one can see, for all time steps t in the set of all time steps T the equation has to hold true. 
This is the constraint for the whole microgrid. The left side of the equation represents the energy supplied by the supplier, the batteries and the main grid, the right side represents the consumers, the batteries while charging and the main grid while exporting energy. If the balance between the consumers and the suppliers is disturbed, severe consequences can happen. Any imbalance results in a change in the frequency. This change may forces more suppliers to stop production because most suppliers can only operate in a narrow frequency range. As a result of this the imbalance ingresses even more. If the change in the frequency exceeds a certain threshold suppliers and consumers still connected to the grid could be damaged. 
The equation itself is compartmentalized into different factors.
On the supply side we have the sum the produced energy in time step t of all suppliers of the system, regardless of wind or solar energy.
We get those values from our model given data for the time step and weather conditions. Our software calculates energy demand forecast and supply forecast based on models of the supplier and consumers. Additionally, the location of the installations are provided by the user and a weather forecast is provided by a weather service. For example, for the wind turbines, the Herman Wobus equation is used to calculate the saturated vapor pressure \cite{NOAA}. This equation is based on empirical data and helps us to incorporate the weather forecast in the wind energy supply forecast. Because all these calculations happen before the optimization step, the solver does not need to know anything about the specifics of the supplier or consumer. This supports reuse-ability of the solver and the models for it. The forecast data is provided as an array of data which contains a value for each time step. The values represent the supply or demand and are measured in Watts. 
The second part is the energy provided by all batteries at a given time step.
The connection to the main grid is resolved in the third part of the supply side of the equation.
It gives the energy provided by the main grid at a given time step t.\\
For the demand side, we chose a similar approach.
We split the equation into several smaller parts.
The first one is the sum over all energy consumers of at a given time step. To calculate the total energy demand. Comparable to the supply forecast, the demand forecast is generated before the optimization step. It is based on the demand profiles described in \cref{subsec:Demand}. The demand is provided as an array of constants to the solver and measured in Watts. Because the supply and the demand are provided as constants to the optimization component, we did not formulate any constraints for the supplier or the consumers.
The second one is the again the part for the batteries.
Here we calculate the energy demand of all batteries at a given time step.
The last part represents the connection to the main grid.
Here we add the energy we provide to the main grid, in other words, what the main grid can demand, at a given time step.\\

There are additional constrains which have to hold true in our simulation. 
For example it is not possible to charge and discharge batteries at the same time.

TODO: Time shifiting, Main grid import and export at the same time? 

shifting
\subsection{Objective of the optimization problem}
To maximize the profit in a specified interval of timesteps $t \in T$ between the time steps $a \in T$ and $b \in T$ we propose the following function:\\
\begin{equation} \label{eq:opt}
\max_{p \in P}{(\sum_{t=a}^{b}{((G_{prod}(t)*price(t))-(G_{demand}(t)*cost(t)))})}
\end{equation}
Where P is th set of all possible decision configurations. $G_{prod}$ and $G_{demand}$ are depending on the decisions made and on the equation for the balance in the microgrid of the exercises before.
\subsection{Explanation Table}

\section{Conclusions}
The power girds of the future will be different from the existing ones. More information will be used to make better decisions, more renewable energy sources will be integrated and more automation will happen. Also, microgrids will be a part of the energy concepts of the future. The simulation system presented in this report could help to gain a better understanding about different aspects of microgirds. The first section provided a motivation and introduction to microgirds. The next section contains basic information which are necessary to understand the problem, the functional requirements and an architecture description. The architecture is split in three layers. The webforntend, the logic layer which contains the simulation and the database layer. It also features a reliable approach to integrate an external weather component.\\

In the future more custom modules could be developed, to integrate different suppliers and consumers.


\printbibliography
%\bibliographystyle{plain}
%\bibliography{bibliography}
All links were checked last on December 13, 2018.
% Ende des Dokuments
\end{document}
