The power girds of the future will be different from the existing ones. More information will be used to make better decisions, more renewable energy sources will be integrated and more automation will happen. Also, microgrids will be a part of the energy concepts of the future. The simulation system presented in this report could help to gain a better understanding about different aspects of microgirds. The first section provided a motivation and introduction to microgirds. The next section contains basic information which are necessary to understand the problem, the functional requirements and an architecture description. The architecture is split in three layers. The webfrontend, the logic layer which contains the simulation and the database layer. It also features a reliable approach to integrate an external weather component. \Cref{sec:Implementation} contains a description of our implementation. \Cref{sec:DemandResponse} provides a description of the optimization problem which has to be solved in order to find the best solution possible for the micogrid. This solution describes when it is profitable to charge batteries, when to shift the demand.\\

\noindent In the future more custom modules could be developed, to integrate even more suppliers and consumers.
