To implement the constraints, we use the library choco solver\footnote{\url{http://www.choco-solver.org/}}. This library helps us to solve the optimization problem for integer variables. As a consequence of this only integer variables are possible and double variables have to be rounded. As described above we would like to use Watts as the unit of measurements for our variables. But this would result in large numbers and a huge range for the variables. To circumvent this, we convert everything to kW and kWh. This helps to reduce the range for each variable and therefore speeds up the solving process. The bounds for the variables are configurable by the user but should remain as low as the specific problem allows. Otherwise, the calculation time and the RAM requirements are very high. To make the calculation of an optimal solution feasible it was necessary to limit the number of consumers to two. This could be one household and one office. Furthermore, only one battery and one supplier is supported. A workaround to these limitations is to combine the loads of different consumers and to combine the supply of different suppliers. For the suppliers there is no difference in the result if all of them get combined before the optimization. Another limitation is that the computation of more than 4 timesteps requires a lot of spare time and depending on the input and the chosen range a lot of RAM.