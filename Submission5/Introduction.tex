\section{Introduction}

The traditional power grid is changing more and more over time.
Due to increasing sensitization for the use of renewable and reliable sources of energy instead of nuclear power sources or fossil energy sources, there is an increasing accommodation of renewable energy.
To fulfill our daily energy need only with such energy sources is quite difficult and needs lots of planning and simulation.
In this work we build a smart energy system to simulate a smart grid.

A smart grid is an energy efficient system with information and communication technology, automation and awareness of energy consumption and supply.
There are many different actors and technologies which are connected to each other and interoperate to optimize the grid.
%TODO Add "`Introduction to smart energy systems"' smart grid slides definitions

A smart energy system creates a bridge between a power grid and a resilient and reliable smart grid.
Users can simulate different energy sources, as well as different kinds of energy consumers, e.g. homes or offices.
Simulation of distributed energy sources and automation of processes build an energy management system with interaction on different levels of the power grid and the consumer grids.
In a smart grid a new aspect is becoming important.
Contrary to currently used normal power grids the exchange of information becomes more and more important.
Especially the consumer side (demand side) of these grids are now being integrated into the power systems in order to ensure optimal power usage and a better understanding of customer needs.
Those new smart grids use various methods to control the flow of energy in order to optimize efficiency and grid usage.
For a fully functional smart grid, the grid has to be made more resistant to any attacks and have mechanisms to self-diagnose problems and take appropriate recovery actions.
The new smart grids offer better ways to integrate renewable but unstable power suppliers like  windturbines and solarpanels through better controllable interactions on a much smaller scale than conventional power grids.
Through these microgrids we can possibly rely completely on renewable energy sources in the future.
If renewable energy sources produce less energy than is needed, the smart grid can buy energy from the main grid.
This process of matching demand and supply can be simulated with our smart energy system.
Using day-ahead energy prices and supply and demand forecasts, the system can simulate demand shifting to be less independent from the main grid.
Therefore our smart energy system can optimize supply, demand, battery data in regard to minimal energy costs with demand shifting and charging/discharging batteries.


The report is structured as follows.
In section \ref{sec:systemdesign} we present our system design.
First, we give some basic foundations which are relevant for our smart energy system, such as the difference between $kWh$ and $kW$.
Afterwards we present our functional requirements for the smart energy systems as user stories.
From these functional requirements we created our architecture which will be described in this section as well. The next section contains the theoretical background to understand the implementation and the description of the basic part of our system. 
In \cref{sec:DemandResponse} we present the mathematical equations the optimizer uses to balance supply and demand in regard to minimal energy costs.
The implementation of the solver is also described in this section.
\Cref{sec:DynamicPricing } shortly describes the day-ahead energy market and our service to collect those prices for the dynamic pricing in our system.
The frontend implementation is described in \cref{sec:frontend}.
In addition to this, we explain which charts we chose to present the data most conveniently to the user.
Finally in \cref{sec:conclusionAndOutlook}, we give a conclusion about the project and present an outlook about possible future work. 